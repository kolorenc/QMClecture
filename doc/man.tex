\documentclass[12pt,a4paper]{article}
\usepackage{lmodern}          % bold tt
\usepackage{listings,xcolor}  % colored code listing
\usepackage{amsmath,amssymb}  % blackboard bold
\usepackage{graphics}

\usepackage[raggedright,small]{titlesec}
%\def\shang{-.8cm}
\def\shang{0pt}
\titlespacing*{\section}{\shang}{*3}{*1}
\titlespacing*{\subsection}{\shang}{*3}{*1}
% \titleformat{name=\section}
%   {\normalfont\large\bfseries}{\hskip\shang\thesection}{.5em}{}{\raggedright}
% \titleformat{name=\section,numberless}
%   {\normalfont\large\bfseries}{\hskip\shang}{0em}{}{}
% \titleformat{name=\subsection}
%   {\normalfont\bfseries}{\hskip\shang\thesubsection}{.5em}{}{}
% \titleformat{name=\subsection,numberless}
%   {\normalfont\bfseries}{\hskip\shang}{0em}{}{}


\IfFileExists{mtpro2.sty}{%
\renewcommand{\rmdefault}{ptm}
\usepackage[lite,subscriptcorrection]{mtpro2}
\linespread{1.04}
}{\relax}

\IfFileExists{microtype.sty}{\usepackage{microtype}}{\relax}

\frenchspacing

\usepackage[square,numbers,sort&compress]{natbib}
%\bibliographystyle{apsrev4-1}
\bibliographystyle{apsrev_mod}
\bibsep=.3em

\usepackage{hyperref}
\hypersetup{  
pdftitle={Basic VMC/DMC code for educational purposes},  
pdfauthor={Jindrich Kolorenc, http://orcid.org/0000-0003-2627-8302},
pdfkeywords={quantum Monte Carlo, diffusion Monte Carlo, Hylleraas
  wave function}
}

%\definecolor{dred}{rgb}{0.4,0.0,0.0}
%\definecolor{dblue}{rgb}{0.0,0.0,0.5}
\definecolor{dred}{rgb}{0.53,0.0,0.0}
\definecolor{dblue}{rgb}{0.14,0.27,0.62}

% I cannot set xleftmargin=\parindent directly because it is not a
% true constant - it is zero after \noindent etc. Therefore I extract
% the value of \parindent here.
\newdimen\xlmargin \xlmargin=\parindent
\lstset{
  basicstyle=\ttfamily\color{dred},
  keywordstyle=\bfseries\color{dblue},
  commentstyle=\color{dgreen},
  columns=flexible,
  language=[95]fortran,
  morekeywords={true,false},
  aboveskip=.5em,
  belowskip=.5em,
  breaklines=true,
  breakatwhitespace=true,
  showspaces=false,
  showstringspaces=false,
  escapechar=\%,
  xleftmargin=\xlmargin
}

% this puts the code inside a minipage so that no page breaks are allowed
\lstnewenvironment{lcode}[1][]%
{
   \vskip0pt\noindent
   \minipage{\linewidth} 
   \lstset{#1}}
{\endminipage\vskip0pt}

\def\code#1{{\lstinline$#1$}}

\newdimen\configwd \configwd=\linewidth
\advance\configwd by -\xlmargin
\def\beginconfig{%
\vskip.8em\nointerlineskip\noindent%
\hskip\xlmargin\begingroup\color{dred}\minipage[t]{\configwd}%
}
\def\endconfig{\endminipage\endgroup\vskip0.8em\noindent}

\newenvironment{arglist}{
 \begin{list}{}{
     \setlength{\topsep}{.6em}
     \setlength{\itemsep}{.7em}
     \setlength{\parsep}{0pt}
     \setlength{\labelwidth}{0pt}
     \setlength{\labelsep}{0pt}
     \setlength{\itemindent}{-1em} % hanging first line of an item
     \setlength{\leftmargin}{1em}
   }}{
 \end{list}}

%-----------------------------------------------------------------------------
% Customization of standard LaTeX article style
\makeatletter

% figure captions
\let\captionsize\small
\newsavebox{\tempbox}
\def\capfraction{1.0}
\makeatletter
\newcommand{\mojec@ption}[2]{
 \captionsize
 %\par\vspace{10pt}
 \vskip\abovecaptionskip
 \sbox{\tempbox}{{\bfseries #1:} #2}
 \ifdim \wd\tempbox > \capfraction\linewidth
  % caption multiline
   \noindent\hbox to\textwidth{\hfill
   \parbox[b]{\capfraction\textwidth}{{\bfseries #1:} #2}
   \hfill}
 \else
   % caption singleline
   \begin{center}#1: #2\end{center}
 \fi
 \normalsize
 \vskip\belowcaptionskip
}
\let\@makecaption\mojec@ption

% no heading for table of contents
\renewcommand\tableofcontents{\@starttoc{toc}}

% header/footer
\def\ps@simplefoot{
\def\@oddhead{\hfill}
\def\@evenhead{\hfill}
\def\@oddfoot{\hfill\thepage}
\def\@evenfoot{\thepage\hfill}
}
\pagestyle{simplefoot}

\makeatother
%-----------------------------------------------------------------------------


% to make the final PDF searchable also for words containing ligatures
% (needs pdfTeX 1.4 and newer)
\IfFileExists{glyphtounicode.tex}{%
\input glyphtounicode \pdfgentounicode=1
}{\relax}

\advance\textheight by 1.4cm
\advance\topmargin by -0.7cm
\advance\textwidth by 0.6cm
\advance\oddsidemargin by -0.3cm

\def\Im{\mathop{\rm Im}\nolimits}
\def\Or{\mathop{\rm O}\nolimits}
\def\rmd{{\rm d}}
\def\rmi{{\rm i}}
\def\rme{{\rm e}}

\begin{document}

\noindent
\hskip\shang{\Large\bfseries Basic VMC/DMC code for educational purposes}
\vskip.5em\noindent
\hskip\shang{\itshape Jind\v rich Koloren\v c, May 2018}
\vskip1em

\hrule
\tableofcontents
\vskip\baselineskip\hrule

\vskip.5em
\noindent

\section{Configuration file}

There is a separate executable for each of the physical systems to be
simulated,
\code{paraHe_run}, \code{othoHe_run}, \code{H2_run} (see below for
details). Each of these
programs performs a VMC simulation followed by a DMC simulation. It
is controlled by a configuration file
\code{qmc_run.cfg}. It contains name--value pairs (the order of these
pairs does not matter). The number of wakers and the number of Monte Carlo
steps are specified as powers of two. The number of walkers is given
as
\beginconfig
\begin{verbatim}
NW        10
\end{verbatim}
\endconfig
which means that there will be $2^\text{\code{NW}}=2^{10}=1024$
walkers (strictly in VMC and on average in DMC).

\subsection{VMC}
The parameters of VMC calculation are specified as
\beginconfig
\begin{verbatim}
VMCtherm  11
VMCsteps  15
VMCtstep   0.1
\end{verbatim}
\endconfig
which defines the number of thermalization/equilibration steps as
$2^\text{\code{VMCtherm}}$ and the number of step when the energy is
measured as $2^\text{\code{VMCsteps}}$. The ``time step''
\code{VMCtstep} has a different meaning for the simple Metropolis
moves and for the diffusion-drift moves. The Metropolis moves are
performed as
\begin{equation}
\mathcal R=\mathcal R'+\xi(\text{\code{VMCtstep}}-0.5)
\end{equation}
where $\xi$ is a uniform random number on the interval $[0,1)$. In the
case of the diffusion-drift moves the meaning is the same as the time
step in DMC. The value of \code{VMCtstep} should to be chosen
such that the acceptance ratio of the proposal moves is not too close
to either 1 or 0.

\subsection{DMC}
The parameters of the DMC simulation are specified analogously,
\beginconfig
\begin{verbatim}
DMCtherm  12
DMCsteps  15
DMCtstep   0.01
DMCruns    1
\end{verbatim}
\endconfig
Setting the number of DMC runs, \code{DMCruns}, larger than 1 is
intended to perform the time-step extrapolation to zero. The code runs
sequential DMC simulations with decreasing time steps
\begin{multline}
\text{\code{DMCtstep}},
\frac{\text{\code{DMCruns}}-1}{\text{\code{DMCruns}}}\,\text{\code{DMCtstep}},
\frac{\text{\code{DMCruns}}-2}{\text{\code{DMCruns}}}\,\text{\code{DMCtstep}},\dots,\\
\frac{1}{\text{\code{DMCruns}}}\,\text{\code{DMCtstep}}\,.
\end{multline}
The data for the extrapolation are written to \code{tstep_extrap.dat}.

\subsection{System-specific data}
Finally, the parametes of the simulated physical system need to be
specified inside the \code{system\{ \}} group. Details are described
in the next section.
 

\section{Available systems}

One can play with three predefined two-electron systems\dots\ or one
can define his/her own by implementing the appropriate
\code{qmc_input} module, sec.~\ref{sec:qmc_input}. 

\subsection{\code{paraHe_run}: helium atom (spin singlet)}

The singlet state of helium atom (or helium-like two-electron
ions) is simulated with the trial wave function having a
Hylleraas(-like) form
\begin{equation}
\Psi_\text{T}(\mathbf r_1,\mathbf r_2)
 =\biggl(1+\frac12\, r_{12}\biggr)\Bigl[1+a\bigl(r_1^2+r_2^2\bigr)\Bigr]
   \,\rme^{-Z(r_1+r_2)}
\end{equation}
with a single variational parameter $a$. It fullfils the electron-ion
and electron-electron cusp conditions for any $a$.

The \code{system\{ \}} group in the configuration file
\code{qmc_run.cfg} has the following content
\beginconfig
\begin{verbatim}
system { Znuc 2.0  a 0.0 }
\end{verbatim}
\endconfig
This example specifies that the nuclear charge is
$Z=\text{\code{Znuc}}=2.0$ (mandatory parameter)
and the variational parameter in the Hyleraas ansatz is
$a=\text{\code{a}}=0.0$. If the parameter $a$ is not given, the code
takes the optimal value that minimizes the total energy.

The total energy corresponding to this $\Psi_\text{T}$ can be
evaluated analytically. It can be written as
\begin{equation}
E=\frac{Z}{4}\,\frac{A}{B}
\end{equation}
where
\begin{align}
A&=
  -3 a^2 \Bigl\{4 Z \bigl[Z (2176 Z+6903)+4992\bigr]-10065\Bigr\}\nonumber\\
 &\qquad -8 a Z^2 \Bigl\{32 Z \bigl[8 Z (3 Z+7)+27\bigr]-567\Bigr\}\nonumber\\
 &\qquad -8 Z^4\Bigl\{4 Z \bigl[Z (16 Z+25)+4\bigr]-35\Bigr\}\,,\\
B&= 9 a^2 \bigl[Z (896 Z+3355)+3680\bigr]\nonumber\\
 &\qquad +24 a Z^2\bigl[Z (64 Z+189)+168\bigr]
  +8Z^4 \bigl[Z (16 Z+35)+24\bigr]
\end{align}
The derivative of the total energy with respect to $a$ is $\partial
E/\partial a=8 Z^3 C/B^2$ where
\begin{align}
C&=9 a^2 \Biggl\{Z \biggl[Z \Bigl[Z \bigl[4096 Z (8 Z+47)+413413\bigr]+304160\Bigr]-43776\biggr]-98910\Biggr\}\nonumber\\
&\quad +12 a Z^2 \Biggl\{2 Z \biggl[Z \Bigl[Z \bigl[8 Z (256 Z+1301)+16329\bigr]-1848\Bigr]-18912\biggr]-18105\Biggr\}\nonumber\\
&\quad -8 Z^4 \biggl\{Z \Bigl[Z \bigl[Z (16 Z+737)+2856\bigr]+3168\Bigr]+1008\biggr\}\,.
\end{align}
To find the optimal $a$ that minimizes the total energy we need to
solve the equation $\partial E/\partial a=0$, which is straightforward
since is is just a quadratic equation (only the final formula is too
long to be explicitly written here).
% \begin{equation}
% a_\text{opt}=\frac{2 \left(\sqrt{Z^4 (2 Z (Z (Z (Z (Z (Z (16 Z (8 Z (512 Z (128
%    Z+1309)+2950393)+56671967)+1264584679)+1064874760)+733280304)+7
%    30902672)+667667520)+327330432)+128388465)}+(18105-2 Z (Z (Z (8
%    Z (256 Z+1301)+16329)-1848)-18912)) Z^2\right)}{3 (Z (Z (Z
%    (4096 Z (8 Z+47)+413413)+304160)-43776)-98910)}
% \end{equation}



\subsection{\code{orthoHe_run}: helium atom (spin triplet)}

The triplet state of the helium atom (or a helium-like two-electron
ion) has an antisymmetric wave function and one can thus investigate
the behavior of the fixed-node constraint. The wave function is
approximated by a Slater determinant constructed from 1s and 2s
orbitals,
\begin{equation}
\Psi_\text{T}(\mathbf r_1,\mathbf r_2)=
  \psi_\text{1s}(r_1) \psi_\text{2s}(r_2)
 -\psi_\text{1s}(r_2) \psi_\text{2s}(r_1)\,,
\end{equation}
where
\begin{equation}
\psi_\text{1s}(r)=\rme^{-Z_1r}
\quad\text{and}\quad
\psi_\text{2s}(r)=\biggl(1-\frac{Z_2 r}{2}\biggr)\rme^{-Z_2 r/2}\,.
\end{equation}
The system is so simple that the nodal hypersurface is entirely
determined by symmetry and it is specified by a condition
$r_1=r_2$. Note that the complete nodal surface is much larger than
the obvious node defined by electron coincidence $\mathbf r_1=\mathbf
r_2$! Our trial function has the exact nodes even though it is
certainly not the exact ground state.

In the \code{system\{ \}} group in the
configuration file \code{qmc_run.cfg} one can specify three parameters
\beginconfig
\begin{verbatim}
system {
  Znuc 2.0
  Z1   1.99364
  Z2   1.55094
}
\end{verbatim}
\endconfig
Only the nuclear charge \code{Znuc} is mandatory, the charges in the
wave function are optional. If they are not given,
\code{Z1}${}={}$\code{Z2}${}={}$\code{Znuc} is used. For VMC one can
treat \code{Z1} and \code{Z2} as variational parameters (the optimal
values are shown in the example above) but for DMC it is better to
satisfy the electron-nucleus cusp conditions by keeping \code{Z1} and
\code{Z2} equal to the charge of the nucleus. For satisfying also the
electron-electron cusp conditions we would need to include a
correlation factor.


\subsection{\code{H2_run}: hydrogen molecule}

The hydrogen molecule is simulated with the Heitler--London trial wave
function
\begin{equation}
\Psi_\text{T}(\mathbf r_1,\mathbf r_2)=
 \rme^{-|\mathbf r_1-\mathbf r_A|-|\mathbf r_2-\mathbf r_B|}
+\rme^{-|\mathbf r_2-\mathbf r_A|-|\mathbf r_1-\mathbf r_B|}\,,
\end{equation}
where $\mathbf r_A$ is the position of one nucleus and $\mathbf r_B$
is the position of the other nucleus. This ansatz does not fulfill the
electron-electron cusp conditions and it has only part of the
electron-nucleus cusps.

The only parameter that can be (and must be) specified in the
\code{system\{ \}} group is the distance between the nuclei
$R=|\mathbf r_A-\mathbf r_B|$,
\beginconfig
\begin{verbatim}
system { R  1.4 }
\end{verbatim}
\endconfig
One can map out the bonding curve by running separate simulations for
different values $R$.

\section{Output}

Apart from the information written to STDOUT, the program writes
several files to the disk.

\subsection{\code{Evmc_trace.dat}}

The average of the total energy across the walker population at each
step of the VMC simulation (the thermalization period is not saved).

\subsection{\code{Edmc_trace.dat}}

The average of the total energy across the walker population at each
step of the DMC simulation (the initial projection period is not
saved).

\subsection{\code{NWdmc_trace.dat}}

The size of the walker population at each step of the DMC simulation
(the initial projection period is not saved).


\subsection{\code{Evmc_reblock.dat}, \code{Edmc_reblock.dat}}

Estimation of the error bar for the total energy in VMC and DMC
simulations by the blocking method -- two neighboring blocks
are merged into one and the number of blocks is thus halved at each
step. The file shows the error bar as a function of the
number of remaining blocks.

\subsection{\code{E2vmc_reblock.dat}, \code{E2dmc_reblock.dat}}

Estimation of the error bar of the variance
\begin{equation}
\frac{%
\langle\Psi_\text{T}|\hat H^2|\Psi_\text{T}\rangle
 - \langle\Psi_\text{T}|\hat H|\Psi_\text{T}\rangle^2
}{%
\langle\Psi_\text{T}|\Psi_\text{T}\rangle
}
\end{equation}
which measures the quality of the trial wave function.

\subsection{\code{tstep_extrap.dat}}

DMC energy as a function of the sime step (if \code{DMCruns}${}>1$).



\section{Source code}

The source code (Fortran) is split to several files. There is a fair
amount of comments in the code so it should be possible to figure out
how the program works.

\subsection{\code{qmc_run.F90}}

The top-level driver that calls the VMC and DMC algorithms on the
given physical system.

\subsection{\code{qmc.F90}}

Implementation of the VMC and DMC methods. The subroutines
\code{vmc_run()} and \code{dmc_run()} are multithreaded.

\subsection{\code{reblocking.f90}}

Implementation of the blocking method.

\subsection{\code{rnd.f90}}

Pseudorandom number generator.

\subsection{\code{paraHe_input.f90}, \code{orthoHe_input.f90},
  \code{H2_input.f90}}
\label{sec:qmc_input}

Calculation of the wave function $\Psi_\text{T}(\mathcal R)$, the local
energy $E_\text{L}=\bigl[\hat H \Psi_\text{T}(\mathcal
R)\bigr]/\Psi_\text{T}(\mathcal R)$ and the drift velocity
$v_\text{D}=\nabla\ln |\Psi_\text{T}(\mathcal R)|$. Each of these
files implements an instance of \code{qmc_input} module.

\subsection{\code{types_const.f90}}

Constants and utility routines.



\bibliography{dft,dmft,lanczos,aim}

\end{document}